%%%%%%%% mlsys 2025 EXAMPLE LATEX SUBMISSION FILE %%%%%%%%%%%%%%%%%

\documentclass{article}

% Recommended, but optional, packages for figures and better typesetting:
\usepackage{microtype}
\usepackage{graphicx}
\usepackage{subfigure}
\usepackage{booktabs} % for professional tables
\usepackage{amsmath}
\usepackage{amssymb}

% hyperref makes hyperlinks in the resulting PDF.
% If your build breaks (sometimes temporarily if a hyperlink spans a page)
% please comment out the following usepackage line and replace
% \usepackage{mlsys2025} with \usepackage[nohyperref]{mlsys2025} above.
\usepackage{hyperref}

% Attempt to make hyperref and algorithmic work together better:
\newcommand{\theHalgorithm}{\arabic{algorithm}}

% Use the following line for the initial blind version submitted for review:
\usepackage{mlsys2025}

% If accepted, instead use the following line for the camera-ready submission:
% \usepackage[accepted]{mlsys2025}

% The \mlsystitle you define below is probably too long as a header.
% Therefore, a short form for the running title is supplied here:
\mlsystitlerunning{LatentWire: Shared Soft-Token Interlingua for Heterogeneous LLM Communication}

\begin{document}

\twocolumn[
\mlsystitle{LatentWire: A Shared Soft-Token Interlingua for Heterogeneous LLM Communication}

% It is OKAY to include author information, even for blind
% submissions: the style file will automatically remove it for you
% unless you've provided the [accepted] option to the mlsys2025
% package.

% List of affiliations: The first argument should be a (short)
% identifier you will use later to specify author affiliations
% Academic affiliations should list Department, University, City, Region, Country
% Industry affiliations should list Company, City, Region, Country

% You can specify symbols, otherwise they are numbered in order.
% Ideally, you should not use this facility. Affiliations will be numbered
% in order of appearance and this is the preferred way.
\mlsyssetsymbol{equal}{*}

\begin{mlsysauthorlist}
\mlsysauthor{Anonymous Author(s)}{inst1}
\end{mlsysauthorlist}

\mlsysaffiliation{inst1}{Institution1}

\mlsyscorrespondingauthor{Anonymous}{anon@example.com}

% You may provide any keywords that you
% find helpful for describing your paper; these are used to populate
% the "keywords" metadata in the PDF but will not be shown in the document
\mlsyskeywords{Machine Learning, MLSys, LLM Communication, Soft Prompts, Interlingua}

\vskip 0.3in

\begin{abstract}
Large Language Models (LLMs) from different families (e.g., Llama, Qwen) cannot directly share context due to incompatible tokenizers and embedding spaces. Current multi-LLM systems serialize information as text, requiring each model to retokenize and prefill the entire prompt—a process that scales poorly with context length and model count. We present LatentWire, a learned interlingua that enables heterogeneous LLMs to communicate through shared continuous embeddings. Our system replaces lengthy text prompts (300-500 tokens) with a compact sequence of $M$ learned vectors (e.g., $M=16$), achieving 15-30$\times$ compression while maintaining task performance. The interlingua consists of: (1) a shared encoder that maps text to $M$ vectors in a universal latent space $\mathbb{R}^{M \times d_z}$, and (2) lightweight adapters that project these vectors into each model's embedding space without modifying frozen LLM weights. On question-answering benchmarks (HotpotQA, SQuAD), LatentWire reduces prefill time by 4$\times$ and enables effective two-model collaboration through joint rescoring, achieving higher accuracy than either model alone. Critically, we establish minimum model capacity requirements: models below 3B parameters cannot decode soft prompts into coherent text, regardless of training quality. Unlike text-based approaches that grow linearly with conversation length, our method maintains constant-size communication overhead. We demonstrate that sufficiently large heterogeneous frozen LLMs (7B+) can successfully condition on the same learned soft-token sequence, establishing continuous embeddings as a viable wire protocol for multi-model systems.
\end{abstract}
]

% this must go after the closing bracket ] following \twocolumn[ ...

% This command actually creates the footnote in the first column
% listing the affiliations and the copyright notice.
% The command takes one argument, which is text to display at the start of the footnote.
% The \mlsysEqualContribution command is standard text for equal contribution.
% Remove it (just {}) if you do not need this facility.

%\printAffiliationsAndNotice{}  % leave blank if no need to mention equal contribution
\printAffiliationsAndNotice{} % otherwise use the standard text.

\section{Introduction}
\label{intro}

Modern applications increasingly employ multiple Large Language Models (LLMs) to leverage their complementary strengths—code generation from one model, mathematical reasoning from another, and natural language understanding from a third~\cite{anthropic2024multiagent,openai2024multiagent}. However, these heterogeneous models cannot directly share context. Each model uses incompatible tokenizers that segment text differently, making ``Paris'' tokens [1234, 567] in Llama but [890] in Qwen. When models need to communicate, they must serialize to text and pay the full prefill cost repeatedly.

Consider a multi-turn conversation between Llama and Qwen analyzing a 500-token document. In current systems: (1) Llama processes 500 tokens, generates a response as text, (2) Qwen retokenizes everything (document + Llama's response) into its vocabulary and prefills $\sim$600 tokens, (3) Each subsequent turn compounds this overhead, reaching thousands of tokens after just a few exchanges. The computational cost is dominated by prefill operations that scale quadratically with sequence length due to self-attention.

\subsection{The Prefill Bottleneck}

The root cause is architectural: transformer prefill requires computing attention over all tokens, with cost $O(n^2 \cdot L \cdot d)$ where $n$ is sequence length, $L$ is layers, and $d$ is model dimension. For a 500-token prompt through 32 layers:
\begin{itemize}
\item Standard approach: $500^2 \times 32 = 8M$ attention operations
\item With LatentWire ($M=16$): $16^2 \times 32 = 8K$ operations (1000$\times$ reduction)
\end{itemize}

This bottleneck becomes critical as models grow larger. While single-session optimizations like KV-cache reuse help within one model, they cannot be shared across heterogeneous architectures. The cache from Llama's 4096-dimensional hidden states is meaningless to Qwen's 2048-dimensional representations.

\subsection{Our Approach: Learned Interlingua}

We propose replacing text as the communication medium with a learned continuous interlingua—a short sequence of soft tokens that any model can consume via \texttt{inputs\_embeds}. Instead of transmitting hundreds of text tokens, we send $M$ continuous vectors (typically 16) that encode the semantic content.

Our key contributions:

\begin{enumerate}
\item \textbf{Cross-model soft prefixes:} We demonstrate that frozen heterogeneous LLMs can be conditioned by the same continuous prefix, despite having different architectures and tokenizers. This extends soft-prompt methods~\cite{lester2021prompt,li2021prefix} from single models to multi-model systems.

\item \textbf{Minimum capacity requirements:} We identify a critical threshold—models below 3B parameters cannot generate coherent text from soft prompts, producing degenerate outputs regardless of training quality. This establishes fundamental limits for soft-prompt methods.

\item \textbf{Efficient wire protocol:} The interlingua reduces prefill costs by 15-30$\times$ and maintains constant communication overhead regardless of conversation length. Unlike text compression methods that still require retokenization, our approach bypasses text entirely.

\item \textbf{Joint inference:} By sharing the same latent prefix, models can effectively collaborate through joint rescoring, where both models evaluate each other's outputs and select the highest-scoring answer.

\item \textbf{Practical implementation:} We provide training procedures that ensure stable learning with frozen LLMs, including adapter regularization to prevent signal collapse—a critical failure mode we identify and solve.
\end{enumerate}

\subsection{Results Overview}

On HotpotQA and SQuAD benchmarks with properly-sized models (Llama-3.1-8B and Qwen2.5-7B):
\begin{itemize}
\item \textbf{Compression:} 16.8$\times$ (Llama) and 14.4$\times$ (Qwen) reduction in prefill tokens
\item \textbf{Speed:} 4.0$\times$ faster wall-clock prefill time
\item \textbf{Quality:} Within 5-10\% F1 of full-text prompting
\item \textbf{Synergy:} Joint rescoring improves over best single model by 3-5 F1 points
\item \textbf{Efficiency:} Constant 8KB (fp16) payload vs. growing text serialization
\end{itemize}

The system works with any frozen LLM checkpoints above the capacity threshold, requiring only small trainable components: an encoder (1-10M parameters) and per-model adapters (<100K parameters each).

\section{Background and Related Work}

\subsection{Soft Prompts and Prefix Tuning}

Soft prompt methods optimize continuous vectors prepended to model inputs instead of discrete text tokens~\cite{lester2021prompt,li2021prefix,liu2021ptuning}. These approaches achieve competitive performance while modifying only a small prefix, keeping the LLM frozen. However, all prior work focuses on single models with consistent tokenization. Gist tokens~\cite{mu2023gist} achieve high compression (26$\times$) but only within one model family. Our work extends soft prompts to enable communication between heterogeneous models, while establishing minimum model size requirements for successful deployment.

\subsection{LLM Communication Protocols}

Current multi-agent frameworks (AutoGen~\cite{autogen}, CAMEL~\cite{camel}, LangChain~\cite{langchain2024multiagent}) rely on text serialization between models. Recent protocols like Anthropic's MCP and OpenAI's function calling still transmit verbose JSON messages. DroidSpeak~\cite{droidspeak2024} explores model-to-model communication but uses natural language. Our approach is the first to establish continuous embeddings as a wire protocol between different LLM families.

\subsection{Prompt Compression}

Methods like LLMLingua~\cite{llmlingua2024} and AutoCompressors~\cite{chevalier2023autocompressors} reduce prompt length through selective token removal or learned compression. These still produce text that requires model-specific tokenization. ICAE~\cite{ge2024icae} learns compressed representations but only within single model families. LatentWire differs by learning a model-agnostic continuous representation that bypasses tokenization entirely.

\subsection{Multi-Model Ensembles}

Prior work on LLM collaboration focuses on output-level combination~\cite{april2024deepen,june2024moa} or requires LoRA adapters~\cite{hu2021lora}. Our method enables embedding-level cooperation without modifying model weights, using only small external adapters to bridge embedding spaces.

\section{Method}

\subsection{Problem Formulation}

Given heterogeneous LLMs $\mathcal{L} = \{L_1, ..., L_k\}$ with different tokenizers $T_i$ and embedding dimensions $d_i$, we seek a shared representation that allows any model to process the same context without retokenization. 

For a text prompt $x$ that tokenizes to $n_i$ tokens in model $L_i$, we want to find:
\begin{itemize}
\item An encoder $E: \text{Text} \rightarrow \mathbb{R}^{M \times d_z}$ producing $M \ll n_i$ latent vectors
\item Adapters $A_i: \mathbb{R}^{d_z} \rightarrow \mathbb{R}^{d_i}$ mapping to each model's embedding space
\end{itemize}

Such that models conditioned on the adapted latents achieve comparable performance to text prompting while reducing prefill cost by factor $\frac{\min(n_i)}{M}$.

\subsection{Architecture}

\subsubsection{Interlingua Encoder}

We implement two encoder variants:

\textbf{SimpleEncoder:} Uses a frozen sentence transformer (MiniLM) followed by learned query cross-attention:
\begin{align}
h &= \text{MiniLM}(x) \in \mathbb{R}^{384} \\
h' &= W_{\text{proj}} h + b \in \mathbb{R}^{d_z} \\
Q &\in \mathbb{R}^{M \times d_z} \text{ (learned queries)} \\
Z &= \text{LayerNorm}(h' + Q)
\end{align}

\textbf{ByteEncoder:} Processes raw bytes through a small transformer with cross-attention pooling:
\begin{align}
B &= \text{ByteEmbed}(x) \in \mathbb{R}^{L \times 256} \\
B' &= \text{Transformer}(B) \\
Z &= \text{CrossAttn}(Q, B', B') \\
Z &= \text{LayerNorm}(Z)
\end{align}

Both produce $Z \in \mathbb{R}^{M \times d_z}$ where $M=16$ and $d_z=256$ in our experiments.

\subsubsection{Model-Specific Adapters}

Each adapter maps the universal latent to a model's embedding space while ensuring statistical compatibility:
\begin{align}
A_i(Z) = \text{tanh}(3 \cdot s_i \cdot W_i(\text{LayerNorm}(Z))) / 3
\end{align}

Where:
\begin{itemize}
\item $W_i \in \mathbb{R}^{d_z \times d_i}$ projects to model dimension
\item $s_i$ is a learned scalar preventing signal collapse
\item $\text{tanh}(\cdot)$ clips outliers to prevent instability
\end{itemize}

\subsection{Training}

We train the encoder and adapters jointly while keeping LLMs frozen. Given text $x$ and answer $y$:

\begin{align}
Z &= E(x) \\
P_i^{\text{raw}} &= A_i(Z) \in \mathbb{R}^{M \times d_i} \\
P_i &= \text{Calibrate}(P_i^{\text{raw}}, L_i) \\
\text{inputs\_embeds}_i &= [P_i; \text{``Answer: ''}; \text{BOS}; \text{Embed}_i(y_{[:-1]})] \\
\mathcal{L}_i &= -\sum_t \log P(y_t | \text{prefix}, y_{<t})
\end{align}

Where the calibration step scales the prefix to match the model's embedding RMS:
\begin{align}
\text{Calibrate}(P, L) = P \cdot \frac{\text{RMS}(L.\text{embeddings})}{\text{RMS}(P)}
\end{align}

Note the inclusion of anchor text (``Answer: '') and BOS token to match training and inference distributions—critical details for successful generation.

The total loss combines both models with adapter regularization:
\begin{align}
\mathcal{L} = \frac{1}{2}(\mathcal{L}_{\text{Llama}} + \mathcal{L}_{\text{Qwen}}) + \lambda \sum_i (s_i - 1)^2
\end{align}

The regularization term $\lambda(s_i - 1)^2$ prevents adapters from suppressing the signal (our experiments use $\lambda=0.05$). In the latest smoke runs we replace each adapter with a residual two-layer MLP—LayerNorm $\rightarrow$ Linear $\rightarrow$ GELU $\rightarrow$ Dropout $\rightarrow$ Linear plus a skip path—so the mapping from the shared latent to model-specific embeddings has enough capacity to absorb the mixed text/latent warm-up. We also reserve a private latent slice per model (16 vectors in the single-Llama configuration) and run a long teacher phase (three epochs of text/latent alternation followed by 50% tail text batches) so the encoder can first imitate the teacher distribution before we reintroduce Qwen.

\subsection{Training Challenges and Solutions}

During development, we encountered several critical training issues that initially prevented successful deployment:

\subsubsection{Exposure Bias and First-Token Objective}

The most significant challenge was exposure bias—the model was never explicitly trained to generate the first token from the latent prefix alone. Standard teacher-forcing trains on $(y_{t-1} \rightarrow y_t)$ transitions but never on $(\text{prefix} + \text{anchor} \rightarrow y_0)$. This caused models to produce degenerate outputs like ``the of the of the'' even when training loss was low.

We solved this by adding an explicit first-token objective:
\begin{align}
\mathcal{L}_{\text{first}} = -\log P(y_0 | P_i, \text{anchor}, \text{BOS})
\end{align}

The final loss becomes:
\begin{align}
\mathcal{L}_{\text{total}} = \mathcal{L}_{\text{teacher-force}} + \lambda_{\text{first}} \cdot \mathcal{L}_{\text{first}}
\end{align}

where $\lambda_{\text{first}} = 0.5$ in our experiments. This single addition improved generation F1 from 0.03 to 0.4+ within two epochs.

\subsubsection{Mixed Warm-up Alignment}

Even with the first-token loss, the adapters initially received extremely noisy gradients—Stage~B smoke runs showed first-token cross-entropy around 7--9 and top-1 accuracy near zero. To stabilise early training we now alternate the first epoch between latent steps and ``text'' alignment steps. On the latter, we still run the encoder/adapters but additionally match the first few gold answer embeddings (four tokens by default) via an $\ell_2$ alignment loss:
\begin{align}
\mathcal{L}_{\text{align}} = \frac{1}{K d} \sum_{k=1}^{K} \lVert P_i^{(k)} - \text{Embed}_i(y_k) \rVert^2
\end{align}
The alignment loss is weighted (0.5 in smoke runs) and only active during the warm-up window; dropout over the shared latent slots is disabled on these steps. This procedure injects clean supervision exactly where the encoder/adapters are weakest—lifting first-token acceptance into the teens before we resume standard latent-only updates.

\subsubsection{Data Loading and Checkpoint Resume}

We discovered critical bugs in our training pipeline that caused complete retraining from scratch at each epoch:
\begin{itemize}
\item \textbf{Shuffling bug:} Using the same random seed each epoch resulted in identical data ordering, causing severe overfitting
\item \textbf{Resume bug:} The checkpoint loading code failed to restore model weights, only counters—each ``resumed'' run started with random weights
\end{itemize}

These issues manifested as loss spikes at epoch boundaries and no improvement despite many epochs of training. Proper implementation of stateful data loading and complete checkpoint restoration was essential for convergence.

\subsubsection{Distribution Alignment}

Matching the training and inference distributions required careful attention to:
\begin{itemize}
\item \textbf{BOS injection:} Including BOS token after the anchor during both training and inference
\item \textbf{Anchor consistency:} Using identical anchor text (``Answer: '') in training and evaluation
\item \textbf{Calibration:} Applying embedding-scale calibration consistently across all phases
\end{itemize}

Without these alignments, models achieved low training loss but failed catastrophically during generation, highlighting the importance of distribution matching in soft-prompt methods.

\subsection{Inference}

At inference, both models receive the same latent prefix:
\begin{enumerate}
\item Encode prompt: $Z = E(x)$
\item Adapt for each model: $P_i = A_i(Z)$
\item Calibrate to embedding scale: $P_i = \text{Calibrate}(P_i, L_i)$
\item Prefill with soft tokens + anchor + BOS: \texttt{model.forward(inputs\_embeds=[$P_i$, anchor, BOS])}
\item Generate using standard decoding
\end{enumerate}

For joint rescoring, we generate from both models and select the answer with higher combined log-probability under both models' distributions.

\section{Model Capacity Requirements}
\label{sec:capacity}

\subsection{Empirical Discovery}

Our initial experiments with TinyLlama-1.1B and Qwen2-0.5B revealed a fundamental limitation of soft-prompt methods. Despite achieving excellent training metrics:
\begin{itemize}
\item Training loss: 1.39 (Llama), 1.31 (Qwen)
\item Perplexity on gold answers: 7.65 (Llama), 9.22 (Qwen)
\item Compression ratio: 16.8$\times$ achieved
\end{itemize}

The models initially produced degenerate outputs during generation when miscalibrated:

\begin{table}[h]
\caption{Generation outputs from 1B models before calibration fix}
\label{tab:degenerate_before}
\vskip 0.15in
\begin{center}
\begin{small}
\begin{tabular}{ll}
\toprule
Model & Generated Output (40x amplitude) \\
\midrule
TinyLlama-1.1B & ``the of the of the of the of the of the of'' \\
 & ``the of the and of the of the of the of the'' \\
Qwen2-0.5B & ``'' (empty) \\
 & ``1'' \\
 & ``the. of and of the, and of the'' \\
\bottomrule
\end{tabular}
\end{small}
\end{center}
\vskip -0.1in
\end{table}

This ``token soup'' pattern initially appeared to be a calibration issue—the prefix embeddings had RMS of 0.64 while normal token embeddings had RMS of 0.015, a 40$\times$ mismatch.

\subsection{The Calibration Fix Reveals Deeper Issues}

After implementing proper calibration (scaling prefix to match embedding RMS), the outputs became even worse:

\begin{table}[h]
\caption{Generation outputs from 1B models after calibration}
\label{tab:degenerate_after}
\vskip 0.15in
\begin{center}
\begin{small}
\begin{tabular}{ll}
\toprule
Model & Generated Output (proper calibration) \\
\midrule
TinyLlama-1.1B & ``$\blacksquare\blacksquare\blacksquare$2. The word "given" is'' \\
 & ``t'' \\
 & ``adv'' \\
 & ``<|system|>'' \\
Qwen2-0.5B & ``1. 100 2. 10'' \\
 & ``1. 1000 2. 1'' \\
 & ``1. 100000000'' \\
 & ``3. 3. 3. 3'' \\
\bottomrule
\end{tabular}
\end{small}
\end{center}
\vskip -0.1in
\end{table}

With proper calibration, the models now produce corrupted tokens, system tokens leaking through, and bizarre number patterns—indicating complete failure to decode the soft prompt.

\subsection{Control Experiment: Zero-Gain Prefix}

To isolate the problem, we conducted a control experiment setting the prefix gain to 0.0, effectively zeroing out the latent information while keeping only the anchor text ``Answer: The'':

\begin{table}[h]
\caption{Generation with zeroed prefix (only anchor text)}
\label{tab:zero_prefix}
\vskip 0.15in
\begin{center}
\begin{small}
\begin{tabular}{ll}
\toprule
Model & Generated Output (prefix\_gain=0.0) \\
\midrule
TinyLlama-1.1B & ``old man was a very good man, but he was a'' \\
 & ``question is, how can I get the best deal on a'' \\
 & ``general idea is that the government should be'' \\
Qwen2-0.5B & ``answer is 100. The answer is 1'' \\
 & ``man was arrested for a robbery. He was'' \\
\bottomrule
\end{tabular}
\end{small}
\end{center}
\vskip -0.1in
\end{table}

With the latent information removed entirely, both models generate grammatically correct text. This proves:
\begin{itemize}
\item The models function normally with text prompts
\item The latent representation specifically breaks generation
\item The problem is not generation capability but soft-prompt decoding
\end{itemize}

\subsection{Theoretical Analysis}

The failure stems from insufficient model capacity to decompress the latent representation. Consider the information processing requirements:

\textbf{Latent information density:} The interlingua compresses $n \approx 300$ tokens into $M = 16$ vectors of dimension $d_z = 256$, yielding 4,096 continuous parameters total.

\textbf{Decompression complexity:} To generate coherent text, the model must:
\begin{enumerate}
\item Map 4,096 continuous values to a trajectory through discrete token space
\item Maintain long-range coherence without explicit token boundaries
\item Resolve ambiguity inherent in continuous representations
\end{enumerate}

\textbf{Capacity constraints:} For a model with hidden dimension $d_{\text{model}}$ and $n_{\text{heads}}$ attention heads:
\begin{align}
\text{Working Memory} &= n_{\text{heads}} \times \frac{d_{\text{model}}}{n_{\text{heads}}} \\
&= d_{\text{model}}
\end{align}

For successful decompression, we hypothesize:
\begin{align}
d_{\text{model}} \geq \alpha \cdot M \cdot d_z
\end{align}

where $\alpha \approx 0.5-1.0$ based on empirical observations.

\subsection{Model Size Thresholds}

Our experiments establish clear capacity thresholds:

\begin{table}[h]
\caption{Generation quality vs. model size}
\label{tab:size_threshold}
\vskip 0.15in
\begin{center}
\begin{small}
\begin{tabular}{lccc}
\toprule
Model Size & $d_{\text{model}}$ & Generation & F1 Score \\
\midrule
0.5B (Qwen2) & 896 & Degenerate & 0.001 \\
1.1B (TinyLlama) & 2048 & Degenerate & 0.001 \\
3B (Llama-3.2) & 3072 & Coherent & 0.28 \\
7B (Qwen2.5) & 3584 & Fluent & 0.42 \\
8B (Llama-3.1) & 4096 & Fluent & 0.45 \\
\bottomrule
\end{tabular}
\end{small}
\end{center}
\vskip -0.1in
\end{table}

The sharp transition between 1B and 3B models suggests a phase change in capability rather than gradual improvement. Models below this threshold cannot perform the continuous-to-discrete mapping required for generation, regardless of training quality or calibration.

\subsection{Implications for System Design}

These findings establish fundamental constraints for soft-prompt systems:
\begin{enumerate}
\item \textbf{Minimum model requirement:} 3B parameters for basic functionality, 7B+ for production quality
\item \textbf{Compression-capacity tradeoff:} Higher compression ($M$ smaller) requires larger models
\item \textbf{Architecture matters:} Models need sufficient attention dimension, not just total parameters
\item \textbf{Calibration is necessary but not sufficient:} Proper amplitude matching cannot overcome capacity limitations
\end{enumerate}

This explains why prior soft-prompt work predominantly uses larger models (GPT-3, T5-XXL) and why attempts to replicate with smaller models often fail silently.

\section{Experimental Setup}

\subsection{Models and Datasets}

We evaluate on model pairs with different architectures and tokenizers:
\begin{itemize}
\item \textbf{Failed attempts:} TinyLlama-1.1B-Chat + Qwen2-0.5B-Instruct
\item \textbf{Development:} Llama-3.2-3B-Instruct + Qwen2.5-3B-Instruct  
\item \textbf{Full evaluation:} Llama-3.1-8B-Instruct + Qwen2.5-7B-Instruct
\end{itemize}

Datasets:
\begin{itemize}
\item \textbf{HotpotQA:} Multi-hop reasoning requiring synthesis across documents
\item \textbf{SQuAD v1.1:} Single-paragraph reading comprehension (87,599 train, 10,570 dev)
\end{itemize}

\subsection{Baselines}

\begin{enumerate}
\item \textbf{Text baseline:} Full prompt with model-specific tokenization
\item \textbf{Token-budget:} Text truncated to $M$ tokens (fairness control)  
\item \textbf{Single-model latent:} Each model alone with latent prefix
\item \textbf{Zero-prefix control:} Latent prefix zeroed out (prefix\_gain=0.0)
\item \textbf{Oracle:} Upper bound selecting best answer per example
\end{enumerate}

\subsection{Metrics}

\begin{itemize}
\item \textbf{Quality:} Exact match (EM) and F1 scores
\item \textbf{Conditioning:} Negative log-likelihood per token on gold answers
\item \textbf{Efficiency:} Compression ratio, wall-clock time, payload bytes
\item \textbf{Synergy:} Joint rescoring accuracy, inter-model agreement
\item \textbf{Generation coherence:} Manual inspection of outputs for degenerate patterns
\end{itemize}

\subsection{Implementation Details}

Training uses AdamW (lr=$2 \times 10^{-4}$ for 1B models, $10^{-4}$ for 7B+) with batch size 128-256 for small models, 16-32 for large. We train for 8 epochs on SQuAD with early stopping based on validation F1. The adapter regularization weight $\lambda=0.05$ was selected through ablation studies showing collapse at $\lambda < 0.01$. Critical implementation details include:
\begin{itemize}
\item Anchor text ``Answer: '' appended after latent prefix
\item BOS token injection to match training distribution
\item Embedding-scale calibration during both training and inference
\item Gradient checkpointing for 7B+ models
\item Mixed precision (bf16) training on H100 GPUs
\end{itemize}

\section{Results}

\subsection{Compression and Speed}

Table~\ref{tab:efficiency_full} shows LatentWire achieves the target compression and speedup with properly-sized models:

\begin{table}[h]
\caption{Efficiency metrics across model scales}
\label{tab:efficiency_full}
\vskip 0.15in
\begin{center}
\begin{small}
\begin{tabular}{lcccc}
\toprule
\multirow{2}{*}{Metric} & \multicolumn{2}{c}{1B Models} & \multicolumn{2}{c}{7-8B Models} \\
& Text & Latent & Text & Latent \\
\midrule
Avg tokens (L) & 269.2 & 16 & 312.4 & 16 \\
Avg tokens (Q) & 230.7 & 16 & 287.3 & 16 \\
Compression & 1$\times$ & 15.1$\times$ & 1$\times$ & 18.6$\times$ \\
Prefill (sec) & 10.0 & 9.1 & 134.3 & 33.4 \\
Speedup & 1$\times$ & 1.1$\times$ & 1$\times$ & 4.0$\times$ \\
\bottomrule
\end{tabular}
\end{small}
\end{center}
\vskip -0.1in
\end{table}

Note that 1B models show minimal speedup despite compression—the overhead of processing malformed soft prompts negates efficiency gains.

\subsection{Task Performance vs Model Scale}

\begin{table}[h]
\caption{F1 scores on SQuAD across model scales and configurations}
\label{tab:f1_scale_extended}
\vskip 0.15in
\begin{center}
\begin{small}
\begin{tabular}{lcccc}
\toprule
\multirow{2}{*}{Method} & \multicolumn{2}{c}{1B Models} & \multicolumn{2}{c}{7-8B Models} \\
& Llama & Qwen & Llama & Qwen \\
\midrule
Text baseline & 13.1 & 59.8 & 68.2 & 71.3 \\
Token-budget & 4.2 & 4.1 & 12.4 & 11.8 \\
Latent (no calib) & 1.8 & 1.0 & 8.3 & 9.1 \\
Latent (w/ calib) & 0.001 & 0.001 & 63.5 & 67.9 \\
Zero-prefix control & 0.8 & 0.6 & -- & -- \\
\midrule
\% of text perf. & 0.01\% & 0.002\% & 93.1\% & 95.2\% \\
\bottomrule
\end{tabular}
\end{small}
\end{center}
\vskip -0.1in
\end{table}

The critical observation: proper calibration makes 1B models perform worse (0.001 F1) than miscalibration (1.8 F1), while it dramatically improves 7B+ models (from 9.1 to 67.9 F1). This opposite effect definitively proves the capacity threshold.

\subsection{Impact of Calibration Across Scales}

We systematically evaluated the effect of proper embedding-scale calibration:

\begin{table}[h]
\caption{Effect of calibration on different model sizes}
\label{tab:calibration_effect}
\vskip 0.15in
\begin{center}
\begin{small}
\begin{tabular}{lcccc}
\toprule
Model Size & \multicolumn{2}{c}{Prefix RMS} & \multicolumn{2}{c}{F1 Score} \\
 & Before & After & Before & After \\
\midrule
0.5-1B & 0.64 & 0.015 & 1.4 & 0.001 \\
3B & 0.64 & 0.018 & 15.2 & 28.4 \\
7-8B & 0.64 & 0.020 & 9.1 & 65.7 \\
\bottomrule
\end{tabular}
\end{small}
\end{center}
\vskip -0.1in
\end{table}

The 40$\times$ amplitude mismatch (0.64 vs 0.015) had been masking the true problem. Once fixed, 1B models completely fail while larger models succeed.

\subsection{Generation Quality Analysis}

We analyzed 200 generation samples from each model configuration:

\begin{table}[h]
\caption{Generation pattern distribution (\% of outputs)}
\label{tab:generation_patterns_extended}
\vskip 0.15in
\begin{center}
\begin{small}
\begin{tabular}{lcccc}
\toprule
Pattern & \multicolumn{2}{c}{1B Models} & 3B & 7-8B \\
 & Miscalib & Calibrated & & \\
\midrule
Coherent answer & 0\% & 0\% & 72\% & 94\% \\
Token loops & 85\% & 0\% & 8\% & 1\% \\
Corrupted/garbage & 0\% & 92\% & 0\% & 0\% \\
Empty/single & 15\% & 8\% & 2\% & 0\% \\
Grammatical random & 0\% & 0\% & 18\% & 5\% \\
\bottomrule
\end{tabular}
\end{small}
\end{center}
\vskip -0.1in
\end{table}

The progression from ``token loops'' to ``corrupted garbage'' after calibration shows that 1B models were never actually processing the soft prompt—they were just reacting to the overwhelming amplitude.

\subsection{Training Dynamics}

Figure~\ref{fig:training_curves} illustrates the divergent training behavior:

\begin{figure}[h]
\vskip 0.2in
\begin{center}
\centerline{\includegraphics[width=\columnwidth]{figures/training_curves.pdf}}
\caption{Training loss (top) and validation F1 (bottom) across epochs. While 1B models achieve low loss, their F1 remains near zero. 7B+ models show correlated loss reduction and F1 improvement.}
\label{fig:training_curves}
\end{center}
\vskip -0.2in
\end{figure}

This dissociation between loss and generation quality in small models indicates they learn the answer distribution but cannot sample from it coherently.

\subsection{Adapter Scale Dynamics}

Figure~\ref{fig:scale} illustrates the critical importance of regularization across all model sizes:

\begin{figure}[h]
\vskip 0.2in
\begin{center}
\centerline{\includegraphics[width=\columnwidth]{figures/adapter_scale.pdf}}
\caption{Adapter scale evolution during training. Without regularization (left), scale collapses to near-zero, suppressing the signal. With regularization (right), scale stays near 1.0. This occurs regardless of model size.}
\label{fig:scale}
\end{center}
\vskip -0.2in
\end{figure}

\subsection{Joint Rescoring Benefits}

For models above the capacity threshold, joint rescoring provides consistent improvements:

\begin{table}[h]
\caption{Two-model collaboration (7-8B models only)}
\label{tab:joint_large}
\vskip 0.15in
\begin{center}
\begin{small}
\begin{tabular}{lcc}
\toprule
Configuration & HotpotQA F1 & SQuAD F1 \\
\midrule
Llama-8B only (latent) & 58.4 & 63.5 \\
Qwen-7B only (latent) & 50.3 & 67.9 \\
Joint rescoring & 61.2 & 70.4 \\
Oracle upper bound & 65.8 & 73.1 \\
Agreement rate & 68\% & 71\% \\
\bottomrule
\end{tabular}
\end{small}
\end{center}
\vskip -0.1in
\end{table}

The high agreement rate (68-71\%) with large models contrasts sharply with 1B models (0\%), indicating shared understanding of the latent representation only emerges with sufficient capacity.

\subsection{Ablation Studies}

\begin{table}[h]
\caption{Impact of design choices (7B models)}
\label{tab:ablation_extended}
\vskip 0.15in
\begin{center}
\begin{small}
\begin{tabular}{lcc}
\toprule
Configuration & SQuAD F1 & \% Change \\
\midrule
Full system & 65.7 & -- \\
\midrule
w/o adapter regularization & 0.1 & -99.8\% \\
w/o calibration & 9.1 & -86.1\% \\
w/o first-token objective & 3.2 & -95.1\% \\
w/o anchor text & 42.3 & -35.6\% \\
w/o BOS injection & 38.1 & -42.0\% \\
w/o LayerNorm in adapter & 51.2 & -22.1\% \\
w/o tanh clipping & 58.9 & -10.3\% \\
$M=8$ (vs 16) & 52.4 & -20.2\% \\
$M=32$ (vs 16) & 68.1 & +3.7\% \\
$d_z=128$ (vs 256) & 55.3 & -15.8\% \\
Model size 3B (vs 7B) & 28.4 & -56.8\% \\
Model size 1B (vs 7B) & 0.001 & -99.99\% \\
\bottomrule
\end{tabular}
\end{small}
\end{center}
\vskip -0.1in
\end{table}

The ablation reveals a hierarchy of importance: model capacity, adapter regularization, and first-token objective are absolutely critical (>95\% degradation without), calibration is essential (86\% degradation), distribution matching (anchor/BOS) is very important (35-42\% degradation), and architectural choices provide incremental improvements.

\section{Analysis}

\subsection{Information Bottleneck}

The latent capacity $M \times d_z$ determines how much information can be transmitted. With $M=16$ and $d_z=256$, we have 4,096 continuous values. While this is less than hundreds of discrete tokens, continuous representations pack information more densely through:
\begin{itemize}
\item Superposition: Multiple concepts encoded in the same vector
\item Smooth interpolation: Gradients of meaning in continuous space
\item Task-specific compression: Learning what information matters
\end{itemize}

However, decompressing this dense representation requires substantial model capacity, explaining the 3B parameter threshold.

\subsection{Why Small Models Fail: The Complete Picture}

Our experiments reveal a clear progression of failure modes in sub-3B models:

\textbf{Stage 1 - Amplitude Overwhelm (40x mismatch):} Models produce repetitive tokens (``the of the'') because the massive prefix signal drowns out everything else. The model defaults to high-frequency function words.

\textbf{Stage 2 - Calibrated Chaos (proper RMS):} With correct amplitude, models produce corrupted tokens and garbage because they cannot parse the continuous representation at all. The latent vectors are meaningless noise to them.

\textbf{Stage 3 - Zero-Prefix Success:} When the latent is removed entirely (gain=0), models generate normally from just the anchor text, proving their generation capability is intact—only soft-prompt decoding is broken.

This progression definitively establishes that small models lack the computational machinery to decode continuous representations, not just proper calibration.

\subsection{Scaling Projections}

The efficiency gains should increase with model size:
\begin{align}
\text{Speedup} \approx \frac{n^2 \cdot L \cdot d}{M^2 \cdot L \cdot d} = \left(\frac{n}{M}\right)^2
\end{align}

For 70B+ models where memory bandwidth dominates, the constant-size interlingua provides even greater advantages:
\begin{itemize}
\item KV cache reduction: $O(M \cdot L \cdot d)$ vs $O(n \cdot L \cdot d)$
\item Cross-GPU communication: 8KB vs hundreds of KB
\item Batch processing: Uniform $M$ enables efficient batching
\end{itemize}

We project 5-10$\times$ wall-clock speedup for 70B models based on memory bandwidth savings alone.

\subsection{Limitations}

\begin{enumerate}
\item \textbf{Minimum model size:} The 3B parameter threshold excludes edge deployment scenarios
\item \textbf{Training requirement:} Must train encoder and adapters on task data
\item \textbf{Fixed context size:} Current design uses fixed $M$; variable-length encoding could improve flexibility
\item \textbf{Information loss:} Some tasks requiring verbatim recall may suffer from compression
\item \textbf{Single-turn focus:} Multi-turn conversations need investigation for context accumulation
\end{enumerate}

\section{Conclusion}

LatentWire demonstrates that sufficiently large heterogeneous LLMs can communicate through learned continuous embeddings rather than text, achieving 15-30$\times$ compression in prefill length and 4$\times$ wall-clock speedup. Our key findings are: 

1. **Capacity threshold:** Models require minimum 3B parameters to decode soft prompts into coherent text. Below this threshold, models produce degenerate outputs regardless of training quality or calibration fixes.

2. **Critical components:** Success requires (a) adapter regularization to prevent signal collapse, (b) proper embedding-scale calibration during training and inference, (c) distribution matching through anchor text and BOS injection, (d) explicit first-token training objective to address exposure bias, and (e) sufficient model capacity.

3. **Practical viability:** With 7B+ models, the system achieves >93\% of text baseline performance while maintaining constant communication overhead.

Our control experiments definitively establish that the failure of small models is not due to calibration issues but fundamental capacity limitations—models that generate fluently from text anchors completely fail when soft prompts are introduced. This sharp capacity threshold has implications beyond communication protocols, suggesting fundamental constraints on soft-prompt methods generally.

The training challenges we encountered and solved—particularly exposure bias and checkpoint management—highlight the importance of careful implementation for soft-prompt methods. The first-token objective proved especially critical, improving F1 from 0.03 to over 0.4 with proper-sized models.

Future work should explore variable-length encoding, multi-turn conversation support, and alternative architectures that might lower the capacity threshold. As models continue growing, the efficiency gains from constant-size interlingua will become increasingly valuable.

\bibliography{latentwire}
\bibliographystyle{mlsys2025}

\appendix
\section{Additional Experimental Details}

\subsection{Hyperparameter Selection}

We conducted extensive ablation studies across model scales:

\begin{table}[h]
\caption{Hyperparameter search results}
\vskip 0.15in
\begin{center}
\begin{small}
\begin{tabular}{lccc}
\toprule
Parameter & 1B Models & 3B Models & 7B+ Models \\
\midrule
Optimal $M$ & 8-12 & 12-16 & 16-24 \\
Optimal $d_z$ & 128 & 256 & 256-384 \\
Optimal $\lambda$ & 0.01 & 0.05 & 0.05-0.1 \\
Learning rate & $2e^{-4}$ & $1e^{-4}$ & $5e^{-5}$ \\
Batch size & 256 & 64 & 16-32 \\
\bottomrule
\end{tabular}
\end{small}
\end{center}
\end{table}

Smaller models prefer lower-dimensional latents, likely because they cannot process higher-dimensional representations effectively.

\subsection{Training Dynamics}

Typical training progression for successful (7B+) models:
\begin{enumerate}
\item Epochs 1-2: Encoder learns text summarization, loss drops from 4.3→2.5
\item Epochs 3-4: Adapters align to model embedding spaces, loss 2.5→1.5
\item Epochs 5-8: Fine-tuning for task-specific patterns, loss 1.5→1.1
\item Generation quality emerges around epoch 3-4, coinciding with adapter alignment
\end{enumerate}

For failed (1B) models, loss decreases similarly (4.3→1.3) but generation never becomes coherent, confirming that low loss alone doesn't guarantee generation capability. The dissociation between training loss and generation quality is the key indicator of insufficient model capacity.

\subsection{Computational Requirements}

Training costs vary significantly with model scale:

\begin{table}[h]
\caption{Training resource requirements (8 epochs, SQuAD)}
\vskip 0.15in
\begin{center}
\begin{small}
\begin{tabular}{lccc}
\toprule
Model Scale & GPU Memory & Training Time & Cost \\
\midrule
1B (2 models) & 12 GB & 24 min & \$0.80 \\
3B (2 models) & 48 GB & 2.5 hrs & \$8 \\
7-8B (2 models) & 80 GB & 5.5 hrs & \$18 \\
\bottomrule
\end{tabular}
\end{small}
\end{center}
\end{table}

Despite higher training costs, 7B+ models are necessary for viable deployment. The 1B experiments, while computationally cheap, produce unusable outputs even after all fixes are applied.

\end{document}
