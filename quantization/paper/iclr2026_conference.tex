\documentclass{article} % For LaTeX2e
\usepackage{iclr2026_conference,times}

% Optional math commands from https://github.com/goodfeli/dlbook_notation.
\input{math_commands.tex}

\usepackage{hyperref}
\usepackage{url}
\usepackage{booktabs}
\usepackage{graphicx}
\usepackage{amsmath,amssymb}
\usepackage{enumitem}

\title{Quantized Cache-to-Cache: Communication-Budgeted KV Transfer for Heterogeneous LLMs}

% Anonymous for submission.
\author{Anonymous Authors}

% \iclrfinalcopy % Uncomment for camera-ready version, but NOT for submission.
\begin{document}

\maketitle

\begin{abstract}
We study communication-efficient transfer between heterogeneous large language models (LLMs) by quantizing Cache-to-Cache (C2C) KV-cache transfer. Our goal is to reduce bandwidth and memory while preserving accuracy. We present post-training quantization (INT8/INT4), cache-length reduction, and accuracy-versus-bytes curves for a heterogeneous model pair. Empirically, quantization is nearly lossless, while cache-length pruning reveals a strong front/back asymmetry that is critical for budgeted transfer. We release a reproducible evaluation pipeline and analysis scripts, and we outline a main-conference path toward sparse, projector-aware token selection and mixed precision.
\end{abstract}

\section{Introduction}
Large language models (LLMs) often communicate via text, which is slow and lossy. Cache-to-Cache (C2C) communicates via KV-cache projection and fusion, but does not address precision or bandwidth constraints. We ask: \emph{How low can KV precision go before accuracy collapses, and can we recover performance under tight communication budgets?}

\paragraph{Contributions.}
\begin{itemize}[leftmargin=*]
  \item We introduce a precision-aware C2C evaluation pipeline and quantify INT8/INT4 PTQ effects on C2C accuracy.
  \item We study cache-length reduction as a second budget axis and show that back-pruning consistently outperforms front-pruning.
  \item We report accuracy vs. communication-budget curves that jointly compare precision and cache length.
  \item We provide a reproducible benchmarking setup and analysis scripts to support extensions to QAT, mixed precision, heterogeneity, and selective transfer.
\end{itemize}

\section{Background and Motivation}
C2C projects sharer KV caches into receiver space and fuses them with learned gates, preserving rich semantics compared to text relay. However, KV caches are large: they scale with sequence length, KV heads, and head dimension. Quantization and cache-length reduction can shrink the communication footprint while retaining accuracy. This work reframes C2C through a communication-budget lens.

\section{Related Work}
\textbf{C2C.} Cache-to-Cache (C2C) enables direct semantic communication by projecting and fusing a sharer model's KV cache into a receiver's KV cache with learnable gates, avoiding intermediate text generation \citep{c2c}.

\textbf{KV communication across agents.} KVComm aligns KV caches across diverging prefixes using training-free offset correction with online anchors \citep{kvcomm}. Q-KVComm adds adaptive layer-wise quantization, hybrid information extraction, and heterogeneous calibration for compressed KV transfer \citep{qkvcomm}. These works focus on multi-agent cache reuse/compression; our work studies quantization and cache-length pruning within the C2C projector+fuser pipeline.

\textbf{Latent collaboration and cache alignment.} KV cache alignment learns a shared latent space with adapters to align KV caches across models \citep{kvalign}. LatentMAS enables latent-space collaboration with shared working memory without extra training \citep{latentmas}. Our approach stays within C2C's KV fusion but emphasizes communication budgets and precision/length tradeoffs.

\textbf{Token selection and KV compression.} Token-level KV selection and value-norm importance improve long-context inference for a single model (ZipCache, TokenSelect, VATP) \citep{zipcache,tokenselect,vatp}. We adopt the budget perspective for C2C rather than single-model KV compression.

\section{Method}
\subsection{C2C Recap}
Let the sharer model produce KV caches $(K^S_\ell,V^S_\ell)$ and the receiver produce $(K^R_\ell,V^R_\ell)$ at layer $\ell$. C2C projects sharer KV into receiver space via $\Pi^K_\ell,\Pi^V_\ell$ and fuses them through a learnable gate:
\[
(K^{R\prime}_\ell,V^{R\prime}_\ell)=\mathcal{F}_\ell\left(K^R_\ell,V^R_\ell,\Pi^K_\ell(K^S_\ell),\Pi^V_\ell(V^S_\ell)\right).
\]
This avoids intermediate text and transfers richer internal semantics.

\subsection{Post-Training Quantization (PTQ)}
We quantize the KV caches using INT8 or INT4/NF4 with per-head scaling. We evaluate accuracy and latency under fixed precision budgets. Our current implementation uses fake-quant (quantize then dequantize) to model quantization noise without bit-packing.

\subsection{Cache-Length Reduction}
We prune KV tokens using a fixed ratio (e.g., 50\%, 25\%, 10\%), reducing transmitted bytes further. We evaluate front-pruning and back-pruning to diagnose which instruction tokens are most valuable for cross-model transfer.

\subsection{Selective and Compressed Cache Transfer (SparseC2C)}
As a main-conference extension, we select a sparse subset of token positions to transfer and fuse. Let $I \subset \{1,\dots,T\}$ be selected tokens and $S_I$ the gather operator. We fuse only selected tokens and scatter updates back:
\[
(\tilde{K}^R_\ell, \tilde{V}^R_\ell) = S_I^\top (K^R_\ell, V^R_\ell),\quad
(\tilde{K}^S_\ell, \tilde{V}^S_\ell) = S_I^\top (K^S_\ell, V^S_\ell)
\]
\[
(\tilde{K}^{R\prime}_\ell, \tilde{V}^{R\prime}_\ell) =
\mathcal{F}_\ell\big(\tilde{K}^R_\ell, \tilde{V}^R_\ell, \Pi^K_\ell(\tilde{K}^S_\ell), \Pi^V_\ell(\tilde{V}^S_\ell)\big).
\]
We then scatter the update to the full cache. We use projector-aware token scoring by computing value norms in receiver space (\texttt{proj\_vnorm\_topk}), tying selection to the cross-model mapping.

\subsection{Communication-Budget Curves}
We report accuracy as a function of transmitted bytes, enabling fair comparison under equal communication constraints. For a sequence of length $T$, the approximate bytes are
\[
{\rm bytes} \approx T \cdot p \cdot 2 \cdot L \cdot H_{kv} \cdot d_h \cdot b/8,
\]
where $p$ is the retained cache proportion, $L$ the number of layers, $H_{kv}$ KV heads, $d_h$ head dim, and $b$ bits per element. We use this accounting for consistent budget curves.

\section{Experiments}
\subsection{Setup}
We evaluate on OpenBookQA and ARC-C with a Qwen3-0.6B receiver and Qwen2.5-0.5B sharer. We follow the C2C eval protocol: temperature 0, max\_new\_tokens 64, no CoT, unified chat template. All models are frozen; only the projector is trained when QAT is enabled. The OpenBookQA test split has 500 samples and ARC-C has 1150 samples.

\subsection{Main Results}
All results below are full runs. PTQ is effectively lossless relative to FP16, and cache pruning shows a strong front/back asymmetry.

\begin{table}[ht]
\centering
\caption{Baseline vs. PTQ (full-cache, \%).}
\begin{tabular}{lcc}
\toprule
Setting & OpenBookQA & ARC-C \\
\midrule
FP16 baseline & 52.8 & 55.1 \\
INT8 PTQ & 52.8 & 55.0 \\
INT4 PTQ & 52.6 & 55.4 \\
\bottomrule
\end{tabular}
\end{table}

\begin{table}[ht]
\centering
\caption{OpenBookQA accuracy (\%, 500 samples) for cache-length pruning (INT8).}
\begin{tabular}{lcccc}
\toprule
Order mode & 75\% & 50\% & 25\% & 10\% \\
\midrule
Front & 44.6 & 43.0 & 38.8 & 38.6 \\
Back  & 52.2 & 52.0 & 50.8 & 49.2 \\
\bottomrule
\end{tabular}
\end{table}

\begin{table}[ht]
\centering
\caption{ARC-C accuracy (\%, 1150 samples) for cache-length pruning (INT8).}
\begin{tabular}{lcccc}
\toprule
Order mode & 75\% & 50\% & 25\% & 10\% \\
\midrule
Front & 40.2 & 46.3 & 38.3 & 40.7 \\
Back  & 55.7 & 57.2 & 56.2 & 53.7 \\
\bottomrule
\end{tabular}
\end{table}

\subsection{Communication-Budget Curve}
Figure~\ref{fig:budget-openbookqa} and Figure~\ref{fig:budget-arc} report accuracy versus effective transmitted bytes. Each point is annotated with the retained cache proportion. These curves provide a single, comparable view across precision (FP16/INT8/INT4) and cache-length reduction.
\begin{figure}[ht]
\centering
\includegraphics[width=0.98\linewidth]{../analysis/m4_budget_curve/budget_curve_openbookqa.png}
\caption{Accuracy vs. communication budget (OpenBookQA).}
\label{fig:budget-openbookqa}
\end{figure}

\begin{figure}[ht]
\centering
\includegraphics[width=0.98\linewidth]{../analysis/m4_budget_curve/budget_curve_arc_c.png}
\caption{Accuracy vs. communication budget (ARC-C).}
\label{fig:budget-arc}
\end{figure}

\subsection{Order-Mode Ablation}
Across all cache lengths, \textbf{back-pruning} (keeping later instruction tokens) consistently outperforms \textbf{front-pruning}. At 50\% cache length, for example, back-pruning retains near-baseline accuracy while front-pruning degrades sharply. This suggests late instruction tokens carry higher utility for cross-model KV fusion, a useful design signal for future selective transfer methods.

\subsection{Main-Conference Extensions (Preliminary)}
We report early results for two main-conference extensions. Mixed precision (INT8 with FP16 in the last layers) remains near baseline across last-2/last-4/last-8 schedules. An alignment-only ablation (same model pair, alignment enabled) reduces accuracy, suggesting alignment should be reserved for heterogeneous pairs.
\begin{table}[ht]
\centering
\caption{Preliminary extension results (\% accuracy).}
\begin{tabular}{lcc}
\toprule
Setting & OpenBookQA & ARC-C \\
\midrule
Mixed precision (INT8 + last-2 FP16) & 53.0 & 55.0 \\
Mixed precision (INT8 + last-4 FP16) & 52.8 & 55.3 \\
Mixed precision (INT8 + last-8 FP16) & 52.4 & 55.2 \\
Alignment ablation (same pair) & 46.8 & 49.6 \\
\bottomrule
\end{tabular}
\end{table}
For SparseC2C (token selection), vnorm/knorm scoring preserves accuracy under aggressive token budgets. At p=0.5, INT8 vnorm achieves 52.4/56.2 (OpenBookQA/ARC-C) while front pruning drops to 44.8/47.6. INT4 vnorm remains strong at 52.0/56.3. Full grids across selection modes (front, random, proj\_vnorm, knorm, vnorm) are reported; heterogeneity runs remain pending.

\begin{table}[ht]
\centering
\caption{SparseC2C token selection at p=0.5 (prompt-only, sparse fuse).}
\begin{tabular}{lcc}
\toprule
Setting & OpenBookQA & ARC-C \\
\midrule
INT8 front & 44.8 & 47.6 \\
INT8 random & 50.0 & 53.0 \\
INT8 proj\_vnorm\_topk & 50.2 & 54.1 \\
INT8 knorm\_topk & 51.8 & 56.3 \\
INT8 vnorm\_topk & 52.4 & 56.2 \\
INT4 front & 44.6 & 47.8 \\
INT4 vnorm\_topk & 52.0 & 56.3 \\
\bottomrule
\end{tabular}
\end{table}

\section{Discussion}
Quantized C2C provides large bandwidth reductions with limited accuracy drop. Cache pruning further improves the tradeoff, suggesting a practical path to deployable multi-LLM communication. A main-conference path includes QAT recovery, mixed-precision schedules, heterogeneous model pairs, and SparseC2C token selection.

\section{Limitations}
Our results currently focus on a single model pair and two datasets. We do not yet report end-to-end latency or FLOP measurements for the fuser, and SparseC2C remains an ongoing extension. These limitations will be addressed in the main-conference track.

\section{Broader Impact}
Communication-efficient multi-LLM systems can reduce compute and latency, but they may also enable higher-throughput deployment of models. We emphasize reproducible evaluation, careful reporting of accuracy/latency tradeoffs, and responsible deployment in sensitive domains.

\section{Conclusion}
We introduce precision-aware C2C and report accuracy vs. bytes curves. This establishes a communication-budget perspective for cross-model KV transfer and opens the door to low-latency, low-bandwidth agent collaboration.

\section*{Acknowledgments}
Placeholder.

\bibliography{iclr2026_conference}
\bibliographystyle{iclr2026_conference}

\end{document}
